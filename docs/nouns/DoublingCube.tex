\section{Doubling Cube}

\subsection{Facts}

\begin{dashed}
    \item The doubling cube controls the stakes of the game.
    \item Games begin with a stake of one, the doubling cube may be used by either player.
    \item Players at the start their turns may propose to double the stakes of the game
          with the doubling cube, the opponent can accept the stakes, or refuse, forcing them
          to concede the game at the current stakes.
    \item The doubling cube multiplies the stakes of the winner at the end of the game.
    \item Redoubling doubles the stakes again and refusing a redouble awards the stake prior to the redouble.
    \item Multiples of 2 up to 64 can appear on the doubling cube.
\end{dashed}

\noindent
\newline\textbf{Doubling Cube} - The doubling cube affects the stakes of the game.
Each game begins at a stake of 1. Using the doubling cube doubles that,
but if the opponent refuses the double then the player who doubled is awarded 1 point.
Subsequent doubles double the stake further until a roof of 64 points is reached.
Refusing a redouble awards the stake prior to the redouble.

\subsection{Summary}
The Doubling Cube will give rise to an attribute and behaviour in \textbf{Game}.

\subsection{Other Notes}

\begin{dashed}
    \item The \textbf{Doubling Cube} multiplier is used by \textbf{Score} to determine the stake awarded to the winning player at the end of each game.
\end{dashed}

